\documentclass[a4paper]{article}
\usepackage{graphicx} % Required for inserting images

\title{Misure di Densit\`a}
\author{Nome Cognome (matricola)\quad
        \vspace{3mm}
        Nome Cognome (matricola)\\
         gruppo A1-1\quad tavolo N}
\date{\today}

\begin{document}

\maketitle
\section{Scopo dell'Esperienza}
In questa esperienza abbiamo misurato la densit\`a di alcuni oggetti attraverso misure dirette della massa e delle dimensioni degli oggetti.

\section{Cenni Teorici}
La densit\`a \`e definita come \ldots.

\section{Materiali e Strumenti di Misura}
%\section{Apparato Strumentale} %da utilizzare quando l'apparato è più complesso
L'esperienza consiste nella misura di masse e volumi di un certo numero di solidi di diverse forme: cilindri, parallelepipedi, sfere e prismi a base esagonale. Per le misure dirette ci siamo serviti di:
\begin{itemize}
    \item calibro ventesimale, risoluzione di \ldots;
    \item calibro cinquantesimale, risoluzione di \ldots;
    \item calibro Palmer, risoluzione di \ldots;
    \item bilancia di precisione, risoluzione di \ldots.
\end{itemize}

\section{Misure Dirette}

Prima di iniziare a prendere le misure, abbiamo verificato \ldots.
Abbiamo poi misurato la massa di ogni pesetto con la bilancia di precisione, 
e le dimensioni utili a calcolare il volume con il calibro ventesimale/cinquantesimale/Palmer, assegnando \ldots come incertezza. I risultati delle misure dirette sono riportati in tabella~\ref{tab:dir_cilindri} per i cilindri, in tabella~\ref{tab:dir_parallelepipedi} per i parallelepipedi, \ldots.

\begin{table}[]
    \centering
    \begin{tabular}{c|c|c}
      $m\pm \sigma_m$ (g)   & $d\pm\sigma_d$ (mm)  & $h\pm\sigma_h$ (mm)\\
      \hline
     $\ldots \pm \ldots$    & $\ldots \pm \ldots$ &$\ldots \pm \ldots$\\
    \end{tabular}
    \caption{Misure dirette dei pesetti a forma di cilindro, dove $h$ \`e l'altezza e $d$ \`e il diametro di base.}
    \label{tab:dir_cilindri}
\end{table}

\begin{table}[]
    \centering
    \begin{tabular}{c|c|c|c}
      $m\pm \sigma_m$ (g)   & $a\pm\sigma_a$ (mm)  & $b\pm\sigma_b$ (mm)& $c\pm\sigma_c$ (mm)\\
      \hline
       $\ldots \pm \ldots$  & $\ldots \pm \ldots$ & $\ldots \pm \ldots$ & $\ldots \pm \ldots$\\
    \end{tabular}
    \caption{Misure dirette dei pesetti a forma di parallelepipedo, dove $a$, $b$, e $c$ sono i tre lati.}
    \label{tab:dir_parallelepipedi}
\end{table}


\section{Analisi Dati}
\subsection{Calcolo dei Volumi}
Per i solidi a forma di cilindro abbiamo usato la seguente formula per calcolare il volume a partire dalle misure dirette riportate in tabella~\ref{tab:dir_cilindri}.
\begin{equation}
    V_c = \ldots
\end{equation}
Dal momento che le misure dirette delle lunghezze sono indipendenti, abbiamo calcolato l'incertezza come \ldots:
\begin{equation}
    \sigma_{V_c} = \ldots
\end{equation}

Per i solidi a forma di \ldots

In tabella~\ref{tab:m_V} sono riportate le masse e i volumi di tutti i solidi.

\begin{table}[]
    \centering
    \begin{tabular}{c|c|c}
      solido   & massa (g)  & volume ($mm^3$)\\
      \hline
       \ldots & $\ldots \pm \ldots$ & $\ldots \pm \ldots$\\
    \end{tabular}
    \caption{Misure di massa e volume per tutti i pesetti.}
    \label{tab:m_V}
\end{table}

\subsection{Fit alla densit\`a}

Prima di effettuare il fit abbiamo costruito il grafico di dispersione di $(m,V)$, riportato in figura~\ref{fig:dispersione},
per individuare i pesetti della stessa densit\`a, ovvero quelli che si dispongono approssimativamente sulla stessa retta.

\begin{figure}
    \centering
     \includegraphics[width=0.5\linewidth]{blankfig.pdf}
    \caption{Grafico di dispersione $(m,V)$.}
    \label{fig:dispersione}
\end{figure}

Abbiamo poi formato \ldots gruppi di pesetti, e per ogni gruppo abbiamo costruito il grafico di
dispersione con le masse sulle ascisse e i volumi sulle ordinate ed effettuato il fit dei minimi
quadrati usando la funzione \verb|curve_fit| della libreria \verb|scipy.optimize| di 
\verb|Python|. Abbiamo utilizzato la funzione:
\begin{equation}
    y = \alpha x + \beta
\end{equation}
da cui si ricava la densit\`a $\rho$ e la sua incertezza:
\begin{equation}
    \rho = \alpha^{-1}
\end{equation}
\begin{equation}
    \sigma_{\rho} = \ldots
\end{equation}

In figura~\ref{fig:densita1} \`e riportato il  fit per il primo gruppo di pesetti, i valori di best fit risultano:
\begin{eqnarray}
    \alpha_1 = (\ldots \pm \ldots)\ \rm [unita'\ di\ misura] \\
    \beta_1 = (\ldots \pm \ldots)\ \rm [unita'\ di\ misura]
\end{eqnarray}
La densit\`a risulta quindi pari a $\rho = (\ldots\pm\ldots)\ \rm [unita'\ di\ misura]$ compatibile/incompatibile con \ldots.

\begin{figure}
    \centering
     \includegraphics[width=0.5\linewidth]{blankfig.pdf}
    \caption{Grafico di dispersione $(m,V)$, best fit per il primo gruppo di pesetti e grafico dei residui.}
    \label{fig:densita1}
\end{figure}

In figura \ldots  \`e riportato il secondo fit, i valori di best fit risultano \ldots

\subsection{Verifica della Legge di Potenze per le sfere}

\ldots

\section{Conclusioni}
In questa esperienza abbiamo misurato la densit\`a di alcuni oggetti attraverso misure dirette della massa e delle dimensioni degli oggetti. Le densit\`a misurate risultano compatibili/incompatibili con i valori tabulati.

\end{document}
